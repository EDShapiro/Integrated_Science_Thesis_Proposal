\documentclass{article}
\usepackage{amsmath}
\usepackage{amssymb}
\usepackage{gensymb}
\usepackage{hyperref}
\usepackage{listings}
\usepackage{graphicx}
\graphicspath{ {c:/Users/EvanD/OneDrive/Pictures/}
\title{Integrated Science Thesis Proposal}
\author{Evan Shapiro \\ Master's of Integrated Science, University of Colorado Denver}
\begin{document}
\maketitle
\tableofcontents
\section{Introduction}
\subsection{Magnetically Confined Fusion}
\subsubsection{Motivation for Studying Plasma Fusion}
The motivation for this project is tied in to the overall goals of the Partnership Center for High-Fidelity Boundary Plasma Simulation, which aims to understand the boundary physics of a magnetically confined plasma in a nuclear fusion reactor using high-fidelity simulations. Per the the project proposal, "The turbulent boundary of fusion plasmas is critical since the physics therein determines the the performance of the reactor core, and the power exhausted through the diverter plates." \cite{PPPL_P:2}
The boundary physics of a magnetically confined plasma within a reactor are tied to variety of parameters. The parameter of interest for this study is the ion diffusivity $ 
As such, a bird's eye view of the current research being performed in the field of plasma physics will be presented. \cite{J_Friedberg:1}
This will be followed by a an overview of the diffusion model that will be explored in this paper.\\
\subsubsection{Magnetically Confined Plasma as an Energy Source}
\subsubsection{Physics of Magnetically Confined Fusion}
\subsubsection{Transport Phenomenon in Fusion}


\newpage

%% Bibliography Start
\bibliographystyle{ieeetr}
\bibliography{Bibli}

\end{document}

