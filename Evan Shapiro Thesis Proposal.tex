\documentclass{article}
\usepackage{amsmath}
\usepackage{amssymb}
\usepackage{gensymb}
\usepackage{hyperref}
\usepackage{listings}
\usepackage{graphicx}
\graphicspath{ {c:/Users/EvanD/OneDrive/Pictures/}}
\title{Integrated Science Thesis Proposal}
\author{Evan Shapiro \\ Master's of Integrated Science, University of Colorado Denver}
\begin{document}
\maketitle
\tableofcontents
\section{Introduction}
\subsection{Magnetically Confined Fusion}
\subsubsection{Motivation for Studying Plasma Fusion}

Nuclear fusion is the 
The motivation for this project is tied in to the overall goals of the Partnership Center for High-Fidelity Boundary Plasma Simulation, which aims to understand the boundary physics of a magnetically confined plasma in a nuclear fusion reactor using high-fidelity simulations. The boundary region in a fusion reactor is defined as “extending 10\% of the outer-minor radius in from the magnetic separatix, through the open field line scrape off layer, out to the material walls.” The separatix is the point where the magnetic field lines cross, which in the case of the Tokamak is at the bottom of of the toroid, while the scrape off layer (SOL) is defined as the plasma region that is characterized by open field lines, and is outside of the separatrix. The SOL absorbs most of the plasma exhaust and transports it along field lines to the divertor plates. The divertor plates are responsible for abosorbing heat and ashe produced by the plasma, minimizing contamination of the plasma, and protecting thermal and neutronic loads.\\

CHECK CITATION – INSERT MAIN MAGNETIC FIELD LINE FIGURE HERE\\

The physics in the boundary region must be better understood before a fully functional fusion reactor can built. A few of the issues related to the plasma boundary physics are outlined below.\\

The projected amount of steady state heat load of the divertor plates in the ITER reactor is projected to be "too severe to be sustainable

If a magnetically confined plasma reaches a heating threshold value, in which the plasma confinement time is significantly enhanced, the plasma transitions from a low-confinement mode (L-Mode) to a high-confinement mode (H-Mode). Once the L-H transition occurs, a steep pedestal in the plasma density develops in the plasma boundary region, as can be seen in figure 2.2. This transition brings a reduction in the radially directed electric field, as well as a reduction in the turbulence intensity, which in turn reduces heat transport, raising the temperature of the plasma core. This transition leads to a 2-3 fold increase in plasma power production, thus the H-mode is the desired operating mode for future fusion reactors.\\



Operating in H-mode requires a stable density pedestal to maintain the raised temperature in the plasma core. However, the steep density gradient “acts as a source of free energy for the magnetohydrodynamic plasma edge localized modes (ELM),” in which the pedestal repeatedly “crashes”, yielding bursts of plasma towards the divertor plates. \\

However, since the high confinement mode is the desired operational mode of the ITER Tokamak fusion reactor, one of the main research goals of the Partnership is to develop a better picture of the physics that occur in this region, and the conditions required to maintain stability of the plasma in H-Mode operation.\\

 \cite{PPPL_P:2}
The boundary physics of a magnetically confined plasma within a reactor are tied to variety of parameters. The parameter of interest for this study is the ion diffusivity 
As such, a bird's eye view of the current research being performed in the field of plasma physics will be presented. \cite{J_Friedberg:1}
This will be followed by a an overview of the diffusion model that will be explored in this paper.\\
\subsubsection{Magnetically Confined Plasma as an Energy Source}
\subsubsection{Physics of Magnetically Confined Fusion}
\subsubsection{Transport Phenomenon in Fusion}
\newpage
%% Bibliography Start
\bibliographystyle{ieeetr}
\bibliography{Bibli}
\end{document}

