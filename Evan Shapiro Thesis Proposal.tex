\documentclass{article}
\usepackage{amsmath}
\usepackage{amssymb}
\usepackage{gensymb}
\usepackage{hyperref}
\usepackage{listings}
\usepackage{graphicx}
\graphicspath{ {c:/Users/EvanD/OneDrive/Pictures/}
\title{Integrated Science Thesis Proposal}
\author{Evan Shapiro \\ Master's of Integrated Science, University of Colorado Denver}
\begin{document}
\maketitle
\tableofcontents
\section{Introduction}
\subsection{Magnetically Confined Fusion}
\subsubsection{Motivation for Studying Plasma Fusion}
The motivation for this project is tied in to the overall goals of the Partnership Center for High-Fidelity Boundary Plasma Simulation, which aims to understand the boundary physics of a magnetically confined plasma in a nuclear fusion reactor using high-fidelity simulations.The project proposal from the Partnership Center for High Fidelity Boundary Plasma Simulations first defines the problem that the partnership is attempting to solve for the ITER Tokama project. The partnership is will be studying the physics of the boundary region of a plasma in a fusion reactor. This region is defined as “extending 10% of the outer-minor radius in from the magnetic separatix, through the open field line scrape off layer, out to the material walls.” The scrape off layer (SOL) is defined as the plasma region that is characterized by open field lines. The SOL absorbs most of the plasma exhaust and transports it along field lines to the divertor plates.”  CHECK CITATION – POSSIBLY INSERT FIGURE HERE
Per the ITER Website ”divertor plates are built into the bottom of the Tokamak to absorb heat and ashe produced by the plasma, minimize plasma contamination of the plasma, and protects the surrounding walls from thermal and neutronic loads.”  Needs citation
The stability of the plasma is dependent on the stability in this region. When the magnetically confined plasma reaches a heating threshold value, the plasma transitions from a low-confinement mode (L-Mode) to a high-confinement mode (H-Mode). The high confinement mode is characterized by a much lower characteristic time over which energy is delivered to the divertor plates. 
Once the L-H transition occurs, a steep pedestal in the plasma density develops in the plasma boundary region. This transition brings a reduction in the radially directed electric field, as well as a reduction in the turbulence intensity, which in turn reduces transport, leading to increased heating of the plasma core. However, this pedestal is highly unstable, and magnetohydrodynamic plasma perturbations can lead to a crash of the pedestal, yielding a high burst of energy which will severely degrade and damage the divertor plates. However, since the high confinement mode is the desired operational mode of the ITER Tokamak fusion reactor, one of the main research goals of the Partnership is to develop a better picture of the physics that occur in this region, and the conditions required to maintain stability of the plasma in H-Mode operation.
 \cite{PPPL_P:2}
The boundary physics of a magnetically confined plasma within a reactor are tied to variety of parameters. The parameter of interest for this study is the ion diffusivity $ 
As such, a bird's eye view of the current research being performed in the field of plasma physics will be presented. \cite{J_Friedberg:1}
This will be followed by a an overview of the diffusion model that will be explored in this paper.\\
\subsubsection{Magnetically Confined Plasma as an Energy Source}
\subsubsection{Physics of Magnetically Confined Fusion}
\subsubsection{Transport Phenomenon in Fusion}


\newpage

%% Bibliography Start
\bibliographystyle{ieeetr}
\bibliography{Bibli}

\end{document}

